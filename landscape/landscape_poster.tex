% Basic preamble
\documentclass[a0paper,landscape,columns=3]{../includes/tex/baposter}

% Pull from header includes


\usepackage[font=small,labelfont=bf]{caption} % specifying captions to tables/figures
\usepackage{booktabs}                         % Horizontal rules in tables
\usepackage{relsize}                          % make text smaller in some places
\usepackage{tipa}
\usepackage{amssymb}
\usepackage{multirow}
\usepackage{multicol}
\usepackage{tikz}
\usepackage[margin=.7cm]{caption}
\usepackage{hhline}
\usepackage{colortbl}


\usetikzlibrary{arrows,decorations.pathmorphing,backgrounds,positioning,fit,matrix,trees,mindmap,shapes}

\graphicspath{{../includes/figs/}} % Directory in which figures are stored

% RU colors
% RU red	RGB 204,0,51

\definecolor{bordercol}{RGB}{0,33,71} % Border color of content boxes
\definecolor{headercol1}{RGB}{204,0,51} % bg color for header in content boxes (left side)
\definecolor{headercol2}{RGB}{204,0,51} % bg color for header in content boxes (right side)
\definecolor{headerfontcol}{RGB}{255,255,255} % Text color for header text in content boxes
\definecolor{boxcolor}{RGB}{255,255,255} % bg color for content in the content boxes
\definecolor{uared}{RGB}{204,0,51}

\usepackage{enumitem}
\setlist{itemsep=.25ex}

% Allow syntax highlighting for r code












\begin{document}




\background{
\begin{tikzpicture}[remember picture,overlay]
  \draw (current page.north west)+(-2em,2em) node[anchor=north west]
  {\includegraphics[height=1.1\textheight, width=1.1\textwidth]{../includes/figs/white}};
\end{tikzpicture}
  }

\begin{poster}{
  grid=false,
  borderColor=bordercol,         % Border color of content boxes
  headerColorOne=headercol1,     % bg color for header in content boxes (l side)
  headerColorTwo=headercol2,     % bg color for header in content boxes (r side)
  headerFontColor=headerfontcol, % Text color for header text in content boxes
  boxColorOne=boxcolor,          % bg color for content in content boxes
  headershape=roundedright,      % Specify rounded corner in content box headers
  headerfont=\Large\sf\bf,       % Font modifiers for text in content box headers
  textborder=rectangle,
  background=user,
  headerborder=closed,           % Set to closed for a line under content box headers
  boxshade=plain
}
{}
%
% TITLE AND AUTHOR NAME -------------------------------------------------------
%
{
 \sf\bf 
 \phantom{.} \\ 
 \vspace{0.2in}
 \LARGE{Title here} \\ 
 \Large{Subtitle here}
}
{
 \vspace{.6em} 
 \textbf{Author1 Name}, 
 \textbf{Author2 Name} \\ 
 \smaller{Rutgers University \\ New Brunswick, NJ, U.S.A.} \\
 {\vspace{-0.4in}\hspace{-10.40in}
  \includegraphics[scale=0.3]{../includes/figs/ru_shield2}\phantom{.}} \\
 {\vspace{-0.20in}\smaller youremail@rutgers.edu} \\
 {\vspace{-0.9in}\hspace{11.05in}
  \includegraphics[scale=0.12]{../includes/figs/rap-group}}\vspace{-.6in}
}



%
% INTRODUCTION ----------------------------------------------------------------
%

\headerbox{Introduction}{name=introduction,column=0,row=.1}{


\vspace{.1in}
\textbf{Present study}:

\begin{itemize} 
  \item This is an itemized list
  \item It can contain as many items as you like
\end{itemize}

\textbf{Previous studies}:  
\vspace{.05in}

References are cited as numbers. Here are a few examples. 
Previous studies found that certain things happen \cite{cho2007prosodically}. 
Another person found that another thing happened \cite{guion2003vowel}, but 
some people don't agree with the results \cite{amengual2012interlingual}.

\begin{itemize}
  \item Itemized lists
  \item Can also be nested
  \begin{itemize}
    \item Like this
    \item and this
    \begin{itemize} 
      \item and this
      \item and this
    \end{itemize}
  \end{itemize}
\end{itemize}

\vspace{.1in}

\textbf{Research Questions}:

\begin{enumerate} 
  \item We can also make numbered lists
  \item Which can also be nested
  \begin{enumerate}
    \item These are good for your hypotheses
    \item another item
  \end{enumerate}
\end{enumerate}

\vspace{1.9in}

}




%
%   MATERIALS AND METHODS -------------------------------------------------------
%

\headerbox{Method}{name=methods,column=1,row=.1}{

\vspace{.1in}
\textbf{Materials}

\begin{itemize}
    \item Materials used
    \item Participants
    \item Stimuli
    \item Etc.
    \item item
\end{itemize}

\vspace{.05in}
\textbf{Procedure}

\begin{itemize}
    \item This is what we did
    \item This is how we did it
    \item So and so did it this way too
    \item Some examples here would be handy
    \item item
\end{itemize}

\vspace{.05in}
\textbf{Analysis}

\begin{itemize}
    \item The data were like this...
    \item This is how we analyzed it.
    \item So and so did this too
    \item This is the best way
    \item Here is an example...
\end{itemize}

\vspace{.1in}

\begin{center}
\captionof{table}{This is a table caption.}
\vspace{.15in}
    \begin{tabular}{@{}cccc@{}}
    \hline \\ [-2ex]
            & Lead            & Short-lag       & Long-lag \\ [1ex] 
    \hline \\ [-2ex]
    English &                 & \textipa{/bdg/} & \textipa{/ptk/} \\ [1ex]
    Spanish & \textipa{/bdg/} & \textipa{/ptk/} & \\ [1ex]
    \hline
    \end{tabular}
\end{center}

\vspace{.375in}

}






%
%   RESULTS ---------------------------------------------------------------------
%

\headerbox{Results}{name=results,column=2,row=.1}{ % To reduce this block to 1 column width, remove 'span=2'

\vspace{.2in}

\begin{center}
    \captionof{figure}{Here is a figure caption. Figure captions can span more than one line. The width and justification is taken care of by the \emph{caption} package (in the preamble).}
    \vspace{.15in}
    \includegraphics[width=.9\textwidth]{placeholder.jpg}
\end{center}

\vspace{.15in}

\begin{center}
\captionof{figure}{You can place two figures side-by-side. Refer one picture (left) and then to the other (right).}
\vspace{.15in}
\includegraphics[width=0.45\linewidth]{placeholder.jpg}
\includegraphics[width=0.45\linewidth]{placeholder.jpg}
\end{center}

\vspace{.15in}

\begin{itemize}
    \item Put some numbers here
    \item (F(1,15) = 14.34; p < 0.05)
    \item Everything was significant
    \item item
    \item item
\end{itemize}

\vspace{.55in}

}





%
%   CONCLUSION ------------------------------------------------------------------
%

\headerbox{Conclusion}{name=conclusion,column=3,row=.1}{

% \vspace{.1in}
\textbf{Summary}

\begin{itemize}
    \item The analysis showed A
    \item The analysis showed B too.
    \item And don't forget about C, which was also important.
\end{itemize}

\textbf{Conclusion}

\begin{itemize}
    \item We learned X, Y, and Z.
    \item This research is in align with A and B
    \item But differs from what this other guy found
    \item Future research should do more things
\end{itemize}

\vspace{.03in}
}








%
%   REFERENCES ------------------------------------------------------------------
%

\headerbox{Selected references}{name=references,column=3,below=conclusion}{

% \smaller % Reduce the font size in this block
\renewcommand{\section}[2]{\vskip 0.05em} % Get rid of the default "References" section title
% \nocite{*} % Insert publications even if they are not cited in the poster

\bibliographystyle{unsrt}
\bibliography{../includes/bib/IEEEabrv,../includes/bib/SpeePros} % Use sample.bib as the bibliography file
\vspace{0.3in}
}


%
%   ACKNOWLEDGEMENTS ------------------------------------------------------------
%

\headerbox{Acknowledgements}{name=acknowledgements,column=3,below=references}{

\vspace{.1in}

% \smaller % Reduce the font size in this block
Somebody probably helped you. Thank them here. :)

\vspace{0.35in}

} 



\end{poster}


\end{document}
