% Basic preamble
\documentclass[a0paper,landscape,columns=2]{../includes/tex/baposter}

% Pull from header includes


\usepackage[font=small,labelfont=bf]{caption} % specifying captions to tables/figures
\usepackage{booktabs}                         % Horizontal rules in tables
\usepackage{relsize}                          % make text smaller in some places
\usepackage{tipa}
\usepackage{amssymb}
\usepackage{multirow}
\usepackage{multicol}
\usepackage{tikz}
\usepackage[margin=.7cm]{caption}
\usepackage{hhline}
\usepackage{colortbl}


\usetikzlibrary{arrows,decorations.pathmorphing,backgrounds,positioning,fit,matrix,trees,mindmap,shapes}

\graphicspath{{../includes/figs/}} % Directory in which figures are stored

% RU colors
% RU red	RGB 204,0,51

\definecolor{bordercol}{RGB}{0,33,71} % Border color of content boxes
\definecolor{headercol1}{RGB}{204,0,51} % bg color for header in content boxes (left side)
\definecolor{headercol2}{RGB}{204,0,51} % bg color for header in content boxes (right side)
\definecolor{headerfontcol}{RGB}{255,255,255} % Text color for header text in content boxes
\definecolor{boxcolor}{RGB}{255,255,255} % bg color for content in the content boxes
\definecolor{uared}{RGB}{204,0,51}

\usepackage{enumitem}
\setlist{itemsep=.25ex}


% Allow syntax highlighting for r code












\begin{document}




\background{
\begin{tikzpicture}[remember picture,overlay]
  \draw (current page.north west)+(-2em,2em) node[anchor=north west]
  {\includegraphics[height=1.1\textheight, width=1.1\textwidth]{../includes/figs/white}};
\end{tikzpicture}
  }

\begin{poster}{
  grid=false,
  borderColor=bordercol,         % Border color of content boxes
  headerColorOne=headercol1,     % bg color for header in content boxes (l side)
  headerColorTwo=headercol2,     % bg color for header in content boxes (r side)
  headerFontColor=headerfontcol, % Text color for header text in content boxes
  boxColorOne=boxcolor,          % bg color for content in content boxes
  headershape=roundedright,      % Specify rounded corner in content box headers
  headerfont=\Large\sf\bf,       % Font modifiers for text in content box headers
  textborder=rectangle,
  background=user,
  headerborder=closed,           % Set to closed for a line under content box headers
  boxshade=plain
}
{}
%
% TITLE AND AUTHOR NAME -------------------------------------------------------
%
{
 \sf\bf 
 \phantom{.} \\ 
 \vspace{0.2in}
 \LARGE{White Hmong Reciprocals and Reflexives}}
{
 \vspace{.6em} 
 \textbf{Quartz Colvin} \\ 
 \smaller{Rutgers University \\ New Brunswick, NJ, U.S.A.} \\
 {\vspace{-0.4in}\hspace{-10.40in}
  \includegraphics[scale=0.3]{../includes/figs/ru_shield2}\phantom{.}} \\
 {\vspace{-0.20in}\smaller quartz.colvin@rutgers.edu} \\
 {\vspace{-0.9in}\hspace{11.05in}
   \includegraphics[scale=0.3]{../includes/figs/ru_shield2}\phantom{.}}
}



%
% INTRODUCTION ----------------------------------------------------------------
%

\headerbox{Introduction}{name=introduction,column=0,row=.1}{

In this project, I provide data from two native speakers to show how reciprocals and reflexives are constructed and bound in White Hmong. I argue that White Hmong follows the ’traditional’ binding theory (\cite{reinhart1983}; \cite{chomsky1986}) and supports that vP and CP are phases in Hmong. I also show that the true reciprocal is a Voice head (not a DP) in Hmong and the `false’ reciprocal is a reflexive construction with dual or plural pronouns.

\vspace{.1in}

Empirically, Hmong has been understudied and most literature of the language are from the 1900s, so this project is a more modern view of how the language is used by immigrant communities and their children in the United States today.
}



%
%   FACTS -------------------------------------------------------
%

\headerbox{Facts}{name=facts,column=0,below=introduction}{

\vspace{.1in}

\begin{itemize}

\item \textbf{Fact 1:} The standard structure of reflexive DPs in Hmong is [Pro + Clf + kheej] (1-2)

\vspace{.1in}

\item \textbf{Fact 2:} kheej ’self’ can occur alone as the object of a clause and bind to the subject (3).

\vspace{.1in}

\item \textbf{Fact 3:} There are two ways to translate reciprocal meanings into Hmong.

\vspace{.1in}

\item[A.] Method 1 is another kheej-reflexive construction using a dual pronoun (5).

\vspace{.1in}

\item[B.] Method 2 is a 'true reciprocal’ where the reciprocal sib is a Voice morpheme and not a nominal (4,6)

\end{itemize}
}




\headerbox{Data}{name=data,column=1,row=.1}{
    \begin{itemize}
        \item[(1)] kuv pom kuv tus kheej \\
            \textsc{1sg} see \textsc{1sg} \textsc{Clf} self \\ 
            \textit{"I see myself."}
        \item[(2)] nws pom nws tus kheej \\
            \textsc{3sg} see \textsc{3sg} \textsc{Clf} self \\
            \textit{"He sees himself"}
        \item[(3)] kuv pom kheej \\
            \textsc{1sg} see self \\
            \textit{"I see myself."}
        \item[(4)] lawv sib txawb pob zeb \\
            \textsc{3pl} \textsc{Recip} throw \textsc{Clf} rock \\
            \textit{"They threw rocks at each other."}
        \item[(5)] nkawm tham txog nkawm tus kheej \\
            \textsc{3du} talk about \textsc{3du} \textsc{Clf} self \\
            \textit{"They (du.) are talking about each other."}
        \item[(6)] nkawm sib tham \\ 
            \textsc{3du} \textsc{Recip} talk \\
            \textit{"They (du.) are talking to each other.}
    \end{itemize}
    
}


\headerbox{Proposal for Reflexives}{name=reflexives,column=1,below=data}{

\vspace{.1in}

\begin{itemize}
\item The clausal projection in Hmong is AspP instead of IP or TP since it doesn’t mark tense and is an analytic language (see \cite{lin2006} for Mandarin)

\vspace{.1in}

\item The subject raises from Spec,vP to Spec,AspP due to a strong EPP on Asp \cite{chomsky1982}.

\item Binding domains are synonymous with phases \cite{chomsky1995}, so binding between the antecedent (in Spec,vP) and the reflexive (the sister of V) happens at the vP phase (7). The reflexive can either be kheej or a full reflexive DP.

\end{itemize}
}



\headerbox{Proposal for Reflexives}{name=reflexives,column=2,row=.1}{

\vspace{.1in}

\begin{itemize}

\item[(7)] ???insert tree (7)

\item I argue that the word sib is a Voice head and not an anaphoric DP

\vspace{.1in}

\item The word order poses no issues for the SVO word order’s syntax.

\vspace{.1in}

\item It’s not novel to associate Voice with reciprocity \cite{kratzer1996}, although
this has never been discussed with Hmong data.

\vspace{.1in}

\item[(8)] ???insert tree (8)

\end{itemize}

}








%
%   RESULTS ---------------------------------------------------------------------
%

\headerbox{Proposal for reciprocals}{name=reciprocals,column=2,below=reflexives}{

\vspace{.1in}

\begin{itemize}

\item The final structure, (9), shows that the SVO word order truly is maintained when we have both sib and a DP object

\item[(9)] ???insert tree (9)

\item The domain of the reciprocal meaning is the phase.

\item Before the subject raises out of the vP structure, it is established as the
agent of the reciprocal action.

\item Not only is the domain of the reciprocal clear, but the timing of the reciprocal mapping is also clear
\end{itemize}

}




%
%   CONCLUSION ------------------------------------------------------------------
%

\headerbox{Conclusion}{name=conclusion,column=3,row=.1}{

% \vspace{.1in}

\begin{itemize}
    \item fill in???
\end{itemize}

\vspace{.03in}
}








%
%   REFERENCES ------------------------------------------------------------------
%

\headerbox{Selected references}{name=references,column=3,below=conclusion}{

% \smaller % Reduce the font size in this block
\renewcommand{\section}[2]{\vskip 0.05em} % Get rid of the default "References" section title
% \nocite{*} % Insert publications even if they are not cited in the poster

\bibliographystyle{unsrt}
\bibliography{../includes/bib/IEEEabrv,../includes/bib/SpeePros} % Use sample.bib as the bibliography file
\vspace{0.3in}
}


%
%   ACKNOWLEDGEMENTS ------------------------------------------------------------
%

\headerbox{Acknowledgements}{name=acknowledgements,column=3,below=references}{

\vspace{.1in}

% \smaller % 

I thank the Hmong consultants Ying and Keng.

\vspace{.1in}

I'd also like to thank Dr. Troy Messick, who taught the class in which I did this project

\vspace{0.35in}

} 



\end{poster}


\end{document}
